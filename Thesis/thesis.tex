\documentclass{amsart}
\usepackage{amsmath,amssymb}
\usepackage{seraf}
\usepackage{tikz}
\usepackage{todonotes}\presetkeys{todonotes}{color=blue!20}{}
\usepackage{bbm}
\usepackage{enumerate}

\title{A Formally Verified Proof of the Central Limit Theorem}
\author{Luke Serafin}
%\date{\today}

\newtheorem{theorem}{Theorem}[section]
\newtheorem{lemma}[theorem]{Lemma}
\newtheorem{corollary}[theorem]{Corollary}
\newtheorem{proposition}[theorem]{Proposition}
\newtheorem*{conjecture*}{Conjecture}
\newtheorem*{theorem*}{Theorem}
\theoremstyle{definition}
\newtheorem{definition}[theorem]{Definition}
\theoremstyle{remark}
\newtheorem*{remark}{Remark}
\newtheorem{example}[theorem]{Example}
\newtheorem*{claim}{Claim}

\newcommand{\bldset}[2]{\{{#1}\mid{#2}\}}
\newcommand{\bldseq}[2]{\langle{#1}\mid{#2}\rangle}
\newcommand{\fim}[2]{{#1}[[{#2}]]}

\begin{document}

\begin{abstract}
We present a formalization of the central limit theorem in the interactive proof assisstant Isabelle.
\end{abstract}

\maketitle

\section{Introduction}

Some worked examples--iterated Bernoulli trials, iterated uniform random selection from (0,1), waiting times in a Poisson process.

Simplified statement of the central limit theorem.

\subsection{History and Overview of Automated and Interactive Deduction}

Developments in logic and computer science leading toward automated deduction.

Davis's implementation of Presburger's decidability procedure.

Newell-Shaw-Simon Logic Theory Machine.

Wang's complete propositional procedure.

Gelernter's geometry machine.

\subsection{History of the Central Limit Theorem and Overview of its Proof}

%%% HISTORY

Ideas leading to formulation of CLT. [Brief history of the normal distribution.]

Initial proof by De Moivre.

Generalizations by Laplace, Cauchy, and others.

Fully general form proved by Lyapunov in 1901.

Modern generalizations and refinements.

Brief history of applications of the CLT. [Quote by Sir Francis Galton.]

%%% OVERVIEW OF PROOF

Measure spaces -- History of use in probability, definition and basic properties, measurable functions.

Almost sure convergence (convergence almost everywhere).

Random variables (as measurable functions). Examples.

Independence of random variables.

Distribution functions -- definition and basic properties.

Weak convergence of distribution functions and of measures (weak* convergence in sense of functional analysis).

Helly selection theorem.

Diagonal method.

Tightness of sequences of measures.

Skorohod's theorem.

Portmanteau theorem.

Characteristic functions (Fourier transforms).

Inversion and continuity theorems (connect to harmonic analysis).

Lindberg CLT.

Lyapounov condition.

\section{Analysis in the Isabelle Interactive Proof Assisstant}

\subsection{Overview of Isabelle}

\subsubsection{Isar}

\subsubsection{Proof Automation Tools}

Auto, simp, force, blast, smt, metis, sledgehammer.

\subsection{Isabelle Locales}

\subsection{Number Systems in Isabelle}

Natural numbers (an inductive type).

Rational numbers.

Real numbers.

Extended real numbers.

Complex numbers.

\subsection{Limits and Continuity}

Conceptual overview of relation between filters, nets, and limits.

Definition and basic properties of filter limits.

Definition and basic properties of continuity.

Overview of limsup and liminf.

Conditionally complete lattices.

Implementation of limsup and liminf.

Supporting automation for limits and continuity.

\subsection{Differentiation}

Fr\'echet differentiation.

\texttt{has\_derivative} / \texttt{f'} dilemma and resolution in Isabelle.

\texttt{has\_derivative at \_ within \_}

Supporting automation for differentiation.

\subsection{Measure Theory}

Implementation of measure spaces (as record types).

$\pi$-$\lambda$ theorem.

Carath\'eodory extension theorem (semirings, rings, and premeasures).

Construction of Lebesgue measure.

Product spaces.

\subsection{Bochner and Lebesgue Integration}

Background on integration theory.

Construction of Bochner and Lebesgue integrals.

\texttt{has\_integral} vs \texttt{integral f = \_} and \texttt{integrable f} tradeoff; resolution in Isabelle (cf. similar dilemma encountered when discussing differentiation).

Lemmata concerning integrability.

Monotone convergence theorem.

Integration over sets and intervals; indicator functions; comparison to differentiation for decision as to whether integral over a set should be fundamental; usefulness and cumbersomeness of $\int_b^a = - \int_a^b$.

Fundamental theorem of calculus.

Fubini's theorem.

Integration of complex-valued functions.

Supporting automation for integration.

Computation of integrals -- sinc in particular.

\section{Proof of the Central Limit Theorem}

The formalization process.

Opportunities for improving process of formalizing mathematical results (in Isabelle in particular).

Opportunities for generalization.

Related projects.

\section{Conclusion}

Situating result in general programme of formal verification of mathematical proofs.

\bibliographystyle{plain}
\bibliography{itp}

\end{document}
