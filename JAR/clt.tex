\documentclass{svjour3}

\usepackage{amsfonts}
\usepackage{amsbsy,amssymb,amsmath}
%\usepackage[latin1]{inputenc}
%\usepackage[T1]{fontenc}
\usepackage{todonotes}
%\usepackage{latexsym}
\usepackage{url}
%\usepackage{listings}
%\usepackage{alltt}
%\usepackage{graphicx}
%\usepackage{color}
%\usepackage{colortbl}
%\usepackage{hyperref}
\usepackage{tikz}
\usetikzlibrary{arrows,positioning,calc,shapes.geometric}
%\usepackage{algorithmicx}
%\usepackage{algpseudocode}
%\usepackage{caption}
%\usepackage{subcaption}
%\newcommand*\Let[2]{\State #1 $\gets$ #2}
%\algrenewcommand\algorithmicrequire{\textbf{Precondition:}}
%\algrenewcommand\algorithmicensure{\textbf{Postcondition:}}
% \linespread{.95}
% \usepackage[scaled=0.80]{beramono}

\newcommand{\RR}{\mathbb{R}}
\newcommand{\QQ}{\mathbb{Q}}
\newcommand{\CC}{\mathbb{C}}
\newcommand{\ZZ}{\mathbb{Z}}
\newcommand{\fn}[1]{\mathtt{#1}} % I used mathtt in the thesis; personally, I think it looks better that way.

% \newcommand{\term}{\mathtt{Term}}
% \newcommand{\sterm}{\mathtt{STerm}}
% \newcommand{\functerm}{\mathtt{FuncTerm}}

% \hypersetup{
%     colorlinks=true,
%     citecolor=blue,
%     linkcolor=blue,
%     urlcolor=blue
% }

\tikzset{
    %Define standard arrow tip
    >=stealth',
    %Define style for boxes
    module/.style={
           rectangle,
           draw=black, very thick,
           text width=7em,
           minimum height=2em,
           text centered},
    % Define arrow style
    interface/.style={
           ->,
           thick,
           shorten <=2pt,
           shorten >=2pt,}
}


\begin{document}


% first the title is needed
\title{A formally verified proof of the Central Limit Theorem}

%\titlerunning{A heuristic prover for real ineq.}  % abbreviated title (for running head)

\author{Jeremy Avigad \and Johannes H\"olzl \and Luke Serafin}
\authorrunning{J.~Avigad, J.~H\"olzl, and L.~Serafin}   % abbreviated author list (for running head)

\institute{J. Avigad \and L. Serafin \at Carnegie Mellon University, Pittsburgh, PA 15213, USA \and J. H\"olzl \at Technische Universit\"at M\"unchen}


\maketitle


\begin{abstract}
We proved the CLT!
%\keywords{nonlinear inequalities, interactive theorem proving}
\end{abstract}


\section{Introduction}
\label{section:introduction}


\section{Overview of the Central Limit Theorem}
\label{section:overview}


Formal statement.

\section{Isabelle and its libraries for analysis}
\label{section:isabelle}

Much of the infrastructure described in this section was available before the project began, but we had to add to it substantially. (We will try to indicate our contribution.)

\subsection{The Isabelle proof assistant}

\subsection{Topology and Limits}

Filter limits (refer to paper)

boundaries, open, closed, compact, \ldots

Introduction rules for open and closed

rule set for continuity

\subsection{Real analysis}

Reals an instance of a conditionally complete lattice, sup and inf.

Derivatives. Frechet.

Integrals (HK, Bochner)

Transcendental functions (exp, sin, \ldots)

\subsection{Complex valued functions}

For Fourier analysis, we need functions from $\RR$ to $\CC$. Derivative: Frechet worked. Integral: initially, pairs, but with Bochner, direct.

\subsection{Measure theory and probability}

Overview, refer to paper

Measures and integrals; dominated convergence and monotone convergence

AE quantifier, rules (AE filter -- use stuff from filters). Fixing countably many values.

densities, push-forward measure, Fubini

Convolutions, independence

Distributions, especially normal distribution (Sudeep Kanav)

Lebesgue-Stieltjes (cdf to measure), cumulative distribution functions

Sets of points of continuity is measurable

\subsection{Calculus exercises}

Sinc, moments of Gaussian distribution

Integration by parts

Change of coordinates

Moments of the normal distribution (Sudeep)

$\sin x / x$ using Fubini (options: polar, contour (have now), or Fubini)

\subsection{Varieties of integrals}

General integral

Set integral

Interval integral.

Substitution, FTC, Manual Eberl strengthened substitution

\subsection{Countability and uncountability}

Reals uncountable (we had to extend to intervals, nonempty open sets).

Closure properties of countable sets

Diagonalization 


\section{The formal proof}
\label{section:formal}

\subsection{Weak convergence}

Definition

Portmanteau theorem

\subsection{Characteristic functions}

Main thing: convolution $\simeq$ pointwise product

Also, approximations

\subsection{Levy inversion}

Used: Si

\subsection{Helly's theorem}

\ldots characterize this as some sort of compactness

\subsection{Levy continuity}


\subsection{The Central Limit Theorem}


\section{Reflections}
\label{section:reflections}

\subsection{Dealing with partial functions}

has-integral, etc.

Fubini, dominated convergence

Message: both representations are useful and needed, but you have to be careful

\subsection{Strategies for limit proofs}

For example, use properties of ordering instead of epsilon delta

\subsection{Strategies for integrals}

affine trick

\subsection{Alternatives}

Stone-Weierstrass, complex analysis countour integrals

\subsection{Cleanup and length}

Originally, combined, 13000

Now, things in this section, 2500

Infrastructure: interval integral, Bochner, set integral

Distributions

give line counts, esp.~for CLT, given infrastructure


cleanup:
\begin{itemize}
 \item moving things to libraries
 \item general refactoring, using general properties rather than fiddly proofs
 \item eliminating duplicated code
 \item choosing good names (esp.~for integrals), rather than ``billingsley 13.1''
\end{itemize}


\end{document}
